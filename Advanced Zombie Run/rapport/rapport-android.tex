\documentclass{article}
\usepackage[utf8]{inputenc}
\usepackage{graphicx}

\title{Advanced ZombieRun}     %% \title est une macro, entre { } figure son premier argument
\author{Tom BESSON\\ Willy FRANÇOIS\\ Jean-Baptiste PERRIN\\ Joris RUBAGOTTI}
\makeindex

\begin{document}


\maketitle


\newpage


\tableofcontents


\newpage


\section{Introduction}


Pour la réalisation d'un projet dans le cadre de la matière de développement d'application nomade, nous avons décidé de créer un jeu intitulé Advanced\\ ZombieRun.

Le jeu consiste en réalisation d'un parcourt d'un point de départ vers un point d'arrivé choisie par le joueur sur une map avec pour détermination de la position de celui-ci l'utilisation du GPS. Mais le jeu génère aléatoirement des Zombies virtuels ayant pour but d'intercepter le joueur et ainsi l'empêcher de gagner la partie.
La condition de victoire est d'arriver sur le point d'arrivé sans avoir été touché par les zombies.

Nous vous présenterons tout au long de rapport les différentes phases du développement, les problèmes rencontrés et les améliorations possibles dans l'avenir.


\section{Les choix de développement}



\section{La structure du projet}


\subsection{HomeActivity}
\subsection{PreferencesActivity}


Cette activité permet de définir les différents paramètres de la partie. On peut y choisir:
\begin{itemize}
\item Notre Nom pour les parties multijoueurs
\item La densité de la population de zombies
\item Leur vitesse de déplacement
\item Notre nombre de points de vie
\item Le type d'alerte en cas de contact
\end{itemize}
Les choix faits par l'utilisateur sont enregistrées dans les SharedPreferences du projet afin d'être récupérées plus tard dans la Map.


\subsection{Map}
\subsection{GameMaster}
\subsection{GameEndActivity}
\subsection{AboutActivity}


\section{Les problèmes rencontrés}


Au cours du développement de l'application, nous avons étés confrontés à plusieurs problèmes.
Le plus gênant de ces problèmes est celui du déplacement des Zombis qui ne se faisait pas comme il faut.
Les différents calculs afin de trouver la position future d'un Zombi en fonction de sa position actuelle,
son angle de direction, et sa distance parcourue nous ont donnés du fil à retordre.

Le Multi-Joueurs a aussi posé quelques soucis.
Le problème étant que nous voulions utiliser le Wi-Fi pour les échanges de données avec le serveur en mode HotSpot,
mais celui ci n'avait pas accès aux clients qui se connectaient à lui.
Nous sommes donc partis sur le BlueTooth, en détresse, vers la fin du projet.


\section{Les améliorations possibles dans l'avenir}


Dans l'avenir, nous comptons rendre fonctionnel le multi-joueurs afin d'offrir une expérience inédite aux utilisateurs de l'application.


\section{Conclusion}

  

\end{document} 